\section{Universal property of the product topology, quotient topology}
06.02

\begin{theorem}
   Let \( X, Y \) be topological spaces.
   The product topology on \( X \times Y \) is the unique topology
   on the product s.t. the following universal property holds
  \begin{equation}
    \label{eq:prod}
    f: Z \to X \times Y \text{ cont. } \iff \pi_X \circ f \text{ and } \pi_Y \circ f \text{ are cont. }
  \end{equation}
\end{theorem}

% https://q.uiver.app/#q=WzAsNCxbMiwwLCJaIl0sWzIsMiwiWCBcXHRpbWVzIFkiXSxbMCwyLCJYIl0sWzQsMiwiWSJdLFsxLDIsIlxccGlfeCIsMl0sWzAsMl0sWzAsM10sWzEsMywiXFxwaV95Il0sWzAsMSwiZiIsMix7InN0eWxlIjp7ImJvZHkiOnsibmFtZSI6ImRhc2hlZCJ9fX1dXQ==
\[\begin{tikzcd}
	&& Z \\
	\\
	X && {X \times Y} && Y
	\arrow[from=1-3, to=3-1]
	\arrow["f"', dashed, from=1-3, to=3-3]
	\arrow[from=1-3, to=3-5]
	\arrow["{\pi_x}"', from=3-3, to=3-1]
	\arrow["{\pi_y}", from=3-3, to=3-5]
\end{tikzcd}\]

\begin{proof}
  We show first that the product topology satisfies
  \eqref{eq:prod} and next that it is the unique
  topology that does so.

  Equip \( X \times Y \) with the product topology.
  \begin{enumerate}
    \item[\( \Rightarrow \))] 
      By function composition.
    \item[\( \Leftarrow \))]
      Let \( U \times V \subseteq X \times Y \) be open.
      Then
      \begin{align*}
        {f}^{-1} (U \times V) &= {f}^{-1} (U \times Y \cap X \times V) \\
                              &= {f}^{-1} (U \times Y) \cap {f}^{-1} (X \times V) \\
                              &= {f}^{-1} ({\pi_X}^{-1}(U) ) \cap {f}^{-1} ({\pi_Y}^{-1} (V) ) \\
                              &= {(\pi_X \circ f)}^{-1} (U) \cap {(\pi_Y \circ f)}^{-1} (V)
      \end{align*}
      And since \( \pi_i \circ f \) are continuous,
      \( {f}^{-1} (U \times V) \) is open. Hence \( f \) is continuous.
  \end{enumerate}

  Now suppose that \( \hat{\tau} \) also satisfies
  \eqref{eq:prod}. We show that
  \( \hat{\tau} \subseteq \tau_{X \times Y} \)
  and that \( \tau_{X \times Y} \subseteq \hat{\tau} \).

  Let \( X \hat{\times} Y \) denote the
  product with \( \hat{\tau} \) as topology.
  Consider the following diagram.
% https://q.uiver.app/#q=WzAsNCxbMiwwLCJYIFxcdGltZXMgWSJdLFswLDIsIlgiXSxbNCwyLCJZIl0sWzIsMiwiWCBcXGhhdHtcXHRpbWVzfSBZIl0sWzAsMSwiXFxwaV9YIiwyXSxbMCwyLCJcXHBpX1kiLDJdLFswLDMsIlxcdGV4dHtpZH0iLDJdLFszLDEsIlxcaGF0e1xccGl9X1giLDJdLFszLDIsIlxcaGF0e1xccGl9X1kiXV0=
\[\begin{tikzcd}
	&& {X \times Y} \\
	\\
	X && {X \hat{\times} Y} && Y
	\arrow["{\pi_X}"', from=1-3, to=3-1]
	\arrow["{\text{id}}"', from=1-3, to=3-3]
	\arrow["{\pi_Y}"', from=1-3, to=3-5]
	\arrow["{\hat{\pi}_X}"', from=3-3, to=3-1]
	\arrow["{\hat{\pi}_Y}", from=3-3, to=3-5]
\end{tikzcd}\]
Since \( \pi_X, \pi_Y \) are continuous,
  and \( \pi_X = \hat{\pi}_X \circ \text{id}, \pi_Y = \hat{\pi}_Y \circ \text{id} \), by \eqref{eq:prod}
  \[
    \text{id}: X \times Y \to X \hat{\times} Y
  \]
  is also continuous. This implies that \( \hat{\tau} \subseteq \tau_{X\times Y} \).

  Now, replace \( X \times Y \) with \( X \hat{\times} Y \):
  % https://q.uiver.app/#q=WzAsNCxbMiwwLCJYIFxcaGF0e1xcdGltZXN9IFkiXSxbMCwyLCJYIl0sWzQsMiwiWSJdLFsyLDIsIlggXFxoYXR7XFx0aW1lc30gWSJdLFswLDEsIlxcaGF0e1xccGl9X1giLDJdLFswLDIsIlxcaGF0e1xccGl9X1kiXSxbMCwzLCJcXHRleHR7aWR9IiwyXSxbMywxLCJcXGhhdHtcXHBpfV9YIiwyXSxbMywyLCJcXGhhdHtcXHBpfV9ZIl1d
\[\begin{tikzcd}
	&& {X \hat{\times} Y} \\
	\\
	X && {X \hat{\times} Y} && Y
	\arrow["{\hat{\pi}_X}"', from=1-3, to=3-1]
	\arrow["{\text{id}}"', from=1-3, to=3-3]
	\arrow["{\hat{\pi}_Y}", from=1-3, to=3-5]
	\arrow["{\hat{\pi}_X}"', from=3-3, to=3-1]
	\arrow["{\hat{\pi}_Y}", from=3-3, to=3-5]
\end{tikzcd}\]
Since \( \text{id}:  X \hat{\times} Y  \to  X \hat{\times} Y  \)
is continuous (by \( \eqref{eq:prod} \) once again),
\( \hat{\pi}_X \) and \( \hat{\pi}_Y \) are continuous.
Take \( U \subseteq X, V \subseteq Y \) opens. Then
\( {\hat{\pi}_X}^{-1}(U) = U \times Y  \) and
\( {\hat{\pi}_Y}^{-1}(V) = X \times V  \) are
both open. Hence
\( U \times V = (U \times Y) \cap (X \times V) \)
is open in \( \hat{\tau} \). Since \( \tau_{X \times Y} \)
is generated by \( \{ U \times V \}  \), we get that
\( \tau_{X \times Y} \subseteq \hat{\tau} \).

We have shown that 
\( \tau_{X \times Y} \subseteq \hat{\tau} \)
and that 
\( \hat{\tau} \subseteq \tau_{X \times Y}\), 
so 
\( \hat{\tau} = \tau_{X \times Y}\).
\end{proof}

\begin{corollary}
   Given \( A, B \) topological spaces.
   Then the universal property says that the map
   \begin{align*}
     \hom_\text{Top} (A, X \times Y) &\longrightarrow \hom_\text{Top} (A, X) \times \hom_\text{Top} (A, Y) \\
     (\varphi : A \to X \times Y) &\longmapsto (\pi_X \circ \varphi : A \to X, \pi_Y \circ \varphi : A \to Y)
  \end{align*}
  exists and is an iso.
\end{corollary}

\subsection{Quotient spaces}

\begin{lemma}
    Let \( \varphi: X \to A \) be a surjection of sets,
    and let \( \sim \) be the equivalence relation given
    by \( x \sim y \iff \varphi(x) = \varphi(y) \).
    Then the induced map 
    \begin{align*}
      \overline{\varphi}: X / \sim &\to A \\
      [x] &\to \varphi(x)
    \end{align*}
    is a bijection.
\end{lemma}

% https://q.uiver.app/#q=WzAsMyxbMCwwLCJYIl0sWzQsMCwiQSJdLFsyLDIsIlgvXFxzaW0iXSxbMCwxLCJcXHZhcnBoaSIsMCx7InN0eWxlIjp7ImhlYWQiOnsibmFtZSI6ImVwaSJ9fX1dLFswLDJdLFsyLDEsIlxcb3ZlcmxpbmV7XFx2YXJwaGl9IiwyXV0=
\[\begin{tikzcd}
	X &&&& A \\
	\\
	&& {X/\sim}
	\arrow["\varphi", two heads, from=1-1, to=1-5]
	\arrow[from=1-1, to=3-3]
	\arrow["{\overline{\varphi}}"', from=3-3, to=1-5]
\end{tikzcd}\]

\begin{proof}
  We first show that \( \overline{\varphi} \) is well defined.
  Consider \( x, y \in [x] \). Then \( x \sim y \), so 
  \( \varphi(x) = \varphi(y) \).
  \( \overline{\varphi} \) is surjective since
  \( \varphi \) is surjective.
  Now, sps. \( \overline{\varphi}([x]) = \overline{\varphi}([y]) \).
  Then
  \( \varphi(x) = \varphi(y) \iff x \sim y \iff [x] = [y] \),
  so \( \overline{\varphi} \) is injective.
  Hence \( \overline{\varphi} \) is both inj. and surj.,
  so it is a bijection.
\end{proof}

\begin{definition}[Quotient topology]
  Let \( X \) be a topological space.
  Let \( \pi: X \to A \) be a surjection.
  The quotient topology on A is formed by declaring
  \( U \subseteq A \) to be open iff. \( {\pi}^{-1} (U) \)
  open in \( X \). 
\end{definition}

\begin{proposition}
  The quotient topology is a topology.
  Moreover, it is the finest topology
  that makes \( \pi \) from above continuous.
\end{proposition}

\begin{proof}
  We show the three properties of a topology.
  \begin{enumerate}
    \item[T1)]
      \( \emptyset \) and \( A \) are both open in \( A \)
    since \( {\pi}^{-1}(\emptyset) = \emptyset  \) and
    \( {\pi}^{-1} (A) = X \) are opens in \( X \).
    \item[T2)]
      Let \( \{ U_i   \}_{i \in I}  \) be a collection
      of opens in \( A \). Then
      \[
        {\pi}^{-1} \left(\bigcup_{i \in I} U_i\right)
        = \bigcup_{i \in I} {\pi}^{-1}  (U_i)
      \]
      is open.
    \item[T3)]
      Let \( \{ V_j   \}_{j \in J}  \) be a finite
      collection of opens in \( A \). Then
      \[
        {\pi}^{-1} \left(\bigcap_{j \in J} V_j\right)
        = \bigcup_{j \in J} {\pi}^{-1}  (V_j)
      \]
      is open.
  \end{enumerate}
  Now suppose \( \hat{\tau} \) is a topology on \( A \)
  such that \( \pi \) is continuous. Then, given \( U \subseteq A \)
  open in \( \hat{\tau} \), 
  \( {\pi}^{-1} (U) \) is open in \( X \),
  so \( U \) is also open in the quotient topology.
  Hence, the quotient topology is the finest topology
  that makes \( \pi \) continuous.
\end{proof}

\begin{definition}
    Given \( \pi: X \to A \) surjective, where \( A \) has the
    quotient topology. Then we call \( \pi \) the quotient map.
\end{definition}

\begin{example}
   Let \( \pi: \mathbb{R} \to \{ a, b, c \} = X \).
   be defined by
   \[
    x \mapsto \begin{cases}
      a & x = 0 \\
      b & x < 0 \\
      c & x > 0
    \end{cases}
   \]
   The inverse images are
   \(
     {\pi}^{-1} (a) = \{ 0 \} ,
    {\pi}^{-1} (b) = (-\infty, 0),
    {\pi}^{-1} (c) = (0, \infty)
   \).
   Hence, the quotient topology is
   \( \tau_X = \{ \emptyset, \{ b \}, \{ c \}, \{ b, c \} \}  \).
\end{example}

\begin{definition}
    We say that a map \( f: X \to Y \) between topological spaces
    is open if for all \( U \subseteq X \) open, then \( f(U) \)
    is open. The map is closed if
    for all \( V \subseteq X \) closed, then \( f(V) \) is
    closed.
\end{definition}

