\section{Homotopy lifting property covering spaces}
28.03

\begin{theorem}
    Let \( p: E \to B \)
    be a covering map and let
    \( e_0 \in E \) such that \( p(e_0) = b_0 \).
    Given a path \( \gamma: I \to B \)
    such that \( \gamma(0) = b_0 \).
    Then there exists a unique lift
    \[
      \tilde{\gamma}: I \to E
    \]
    such that \( \tilde{\gamma}(0) = e_0 \).
\end{theorem}

\begin{proof}
    technical. todo.
\end{proof}

\begin{theorem}[Homotopy lifting property]
  \label{thm:hom_lift}
   Let \( p: E \to B \)  be a covering map
   and let \( H: I \times I \to B \)
   such that \( H(0, 0) = b_0 \), and
   let \( e_0 \in E \) such that
   \( p(e_0) = b_0 \).
   Then there exists a unique lift
   \( \tilde{H}: I \times I \to E \)
   such that \( \tilde{H}(0, 0) = e_0 \).
   Moreover, if \( H \) is a path homotopy
   then so is \( \tilde{H} \).
\end{theorem}

\begin{proof}
    technical. todo.
\end{proof}

\begin{proposition}
    Let \( p: E \to B \) be a covering map.
    Let \( e_0 \in E \) such that \( p(e_0) = b_0 \).
    Then there exists an assignment
    \begin{align}
      \pi_1(B, b_0) &\longrightarrow  {p}^{-1} (b_0) \\
      \gamma &\longmapsto \tilde{\gamma} (1)
    \end{align}
\end{proposition}

\begin{proof}
  \( \tilde{\gamma}(1) \in {p}^{-1} (b_0) \) since
  \[
    p(\tilde{\gamma}(1)) = \gamma(1) = \gamma(0) = b_0
  \]
  To show that the assignment
  is well defined we pick two
  path homotopic maps \( \gamma_1 \simeq_p \gamma_2 \).
  We need to show that
  \( \tilde{\gamma_1}(1) = \tilde{\gamma_2}(1)\).
  Let \( H: I \times I \to B \) be a
  path homotopy of \( \gamma_1 \) and 
  \( \gamma_2 \).
  Theorem \ref{thm:hom_lift}
  gives a unique path homotopy
  \[
    \tilde{H}: I \times I \to E
  \]
  Note that
  \begin{align*}
    \tilde{H}(0, s) = \tilde{\gamma_1} \\
    \tilde{H}(1, s) = \tilde{\gamma_2} \\
  \end{align*}
  Since \( \tilde{H} \) is a path homotopy
  we have that \( \tilde{H}(t, 1) \) is
  constant, so
  \[
    \tilde{H}(0, 1) = \tilde{\gamma_1}(1), 
    \tilde{H}(1, 1) = \tilde{\gamma_2}(1)
  \]
  This means that \( \tilde{\gamma_1}(1)
  = \tilde{\gamma_2}(1) \).

\end{proof}

\begin{definition}[Lifting correspondence]
   The assignment
   \[
    \pi_1(B, b_0) \to {p}^{-1} (b_0)
   \]
   is called the lifting correspondence.
\end{definition}

\begin{theorem}
    Let \( p: E \to B \) be a covering map.
    Then the lifting correspondence is
    \begin{enumerate}
      \item surjective if \( E \) is path connected.
      \item bijective if \( E \) is simply connected.
    \end{enumerate}
\end{theorem}

\begin{proof}
    Let \( e_0 \in {p}^{-1} (b_0) \).
    Since \( E \) is path connected I 
    can pick a path
    \[
      \gamma: I \to E
    \]
    from \( e_0 \) to \( e_1 \).
    \( \gamma \) is a lift of
    \( p \circ \gamma \).
    Hence \( (\widetilde{p \circ \gamma})(1) = \gamma(1) = e_1 \).
    Hence, the lifting correspondence is surjective if
    \( E \) is path connected.
    
    Let \( \gamma_1, \gamma_2 \in \pi_1(B, b_0) \) such that
    \( \tilde{\gamma}_1(1) = \tilde{\gamma}_2(1) = e_1 \).
    Consider the loop \[ {\tilde{\gamma}_2}^{-1} * \tilde{\gamma}_1 \]
    Since \( {\tilde{\gamma}_2}^{-1} * \tilde{\gamma}_1 \) is a loop
    and \( E \) is simply connected it is homotopic to the constant loop
    at \( e_1 \).
    This implies that
    \[
      p (\tilde{\gamma}_2^{-1} * \tilde{\gamma}_1) = \gamma_2^{-1} * \gamma_1
    \]
    is homotopic to the constant path at \( b_0 \).
    Hence \( \gamma_1 \simeq_p \gamma_2 \), so
    the lifting correspondence is also injective,  
    and thus bijective.
\end{proof}

\begin{corollary}
  \label{cor:S1_Z_bij}
    There exists a bijection
    \[
      \pi_1(\mathbb{S}^1, *) \leftrightarrow \mathbb{Z}
    \]
\end{corollary}

Next time we show that this bijection
is a group homomorphism, and
hence a group isomorphism.
Also: Brouwer fixed point theorem
and the fundamental theorem of algebra.
