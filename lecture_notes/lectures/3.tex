\section{Topological spaces: First definitions and examples}
16.01

\subsection{Topological spaces}

\begin{definition}[Topological space]
  \label{def:top_space}
  A topological space is a pair \( (X, \tau) \),
  where \( X \) is a set and \( \tau \subseteq \mathcal{P}(X) \)
  is a collection of subsets of \( X \).
  We call the elements in \( \tau \) the open sets in \( X \).
  \( \tau \) satisfies

  \begin{enumerate}
    \item[T1)] \( \emptyset, X \subseteq \tau \).
    \item[T2)] Given a collection \( \{U_i\}_{i \in I} \) of
      open sets, then \( \bigcup_{i\in I} U_i\) is open.
    \item[T3)] Given a finite collection \( \{V_j\}_{j \in J} \), \( \abs{J} < \infty \) of open sets, then \( \bigcap_{j\in J} V_j\) is open.
  \end{enumerate}
\end{definition}

\begin{proposition}
   Let \( (X, d) \) be a metric space. Then \( X \) is a topological space
   with \( \tau = \{ U \subseteq X \mid \forall x \in U, \exists \varepsilon > 0 \text{ s.t. } B(x, \varepsilon) \subseteq U \}  \).
\end{proposition}

\begin{proof} We show that the three axioms in \ref{def:top_space} are satisfied.
   \begin{enumerate}
     \item[T1)] Trivially true.
     \item[T2)] Given \( \{ U_i \}_{i \in I}  \) such that \( U_i \subseteq \tau \) for all \( i \in I \).
       Take some \( x \in \cup_{i\in I} U_i \). Then there exists some \( i_0 \) such that
       \( x \in U_{i_0} \).
       Since \( U_{i_0} \) is open there exists \( \varepsilon_{i_0} > 0 \) such that
      \[
        B(x, \varepsilon_{i_0}) \subseteq U_{i_0} \subseteq \bigcup_{i \in I } U_i.
      \] 
    \item[T3)] Given a finite index set \( J \) and \( \{ V_j  \}_{j \in J}  \) such that \( V_j \in \tau \) for all \( j \in J \).
      Let \( x \in \cap_{j \in J} V_j \). Then \( x \in V_j \) for all \( j \in J \).
      Let \( \varepsilon_j > 0 \) be such that \( B(x, \varepsilon_j) \subseteq V_j \).
      Define \( \varepsilon = \min_j \varepsilon_j \).
      Then
      \[
        B(x, \varepsilon_j) \subseteq \bigcap_{j \in J} V_j.
      \]
   \end{enumerate}
\end{proof}

\begin{example}
   It is important that \( J \) is finite. 
   Take \( X = \mathbb{R} \).
   Let \( V_n = (-1/n, 1/n) \). Then 
   \[
     \bigcap_{n \in \mathbb{N}} V_n = \{ 0 \} \notin \tau.
   \]
\end{example}

\begin{example}
  Let \( X \) be a set and let \( \tau_{dis} = \mathcal{P}(X)\).
  \( \tau_{dis} \) is a topology on \( X \).
\end{example}

\begin{example}
  Let \( X \) be a set and let \( \tau_{ind} = \{ \emptyset, X \}  \).
  \( \tau_{ind} \) is a topology on \( X \).
\end{example}

\begin{example}
  Given \( (X, \tau) \), then \( (X, \tau) = (X, \tau_{dis}) \)
  iff. \( \{ x\}  \) is open \( \forall x \in X \).
\end{example}

\begin{example}
    Let \( X \) be a set and recall the discrete metric \( \delta_X: X \times X \to \mathbb{R} \).
    Then \( X \) is equal to \( (X, \tau_{dis}) \) as a topological space.
\end{example}

\begin{example}
    Let \( X \) be a set, and declare \( U \subseteq X \) to be open
    if \( X \setminus U \) is finite.
    We call the collection \( \tau_{cof} \) the cofinite topology.
\end{example}

\begin{example}
  Let \( \hat{\mathbb{N}} = \mathbb{N} \cup \{ \infty \} \).
  \( U \subseteq \hat{\mathbb{N}} \) is open if either
  \( \infty \notin U \) or \( \infty \in U \) and \( U^\mathsf{c} \) is finite.
\end{example}

\begin{definition}[Neighbourhood]
   Let \( (X, \tau) \) be a topological space.
   Then \( U \subseteq X \) is a neighbourhood
   if \( x \in U \) and \( U \) is open.
\end{definition}

\begin{theorem}
   Let \( (X, \tau) \) be a topological space.
   \( U \subseteq X \) is open iff. \( \forall x \in U \)
   there exists a nbh.  \( V_x \)
   of \( x \) such that \( V_x \subseteq U \).
\end{theorem}

\begin{proof} We show both directions.
  \begin{enumerate}
    \item[\( \Rightarrow \))] Assume \( U \) open. Then for all \( x \in U \)
      \( U \) is a nbh. of \( x \) and \( U \subseteq U \).
    \item[\( \Leftarrow \))] Assume that \( \forall x \in U \) there exists
      nbhs. \( V_x \) such that \( V_x \subseteq U \). Then
      \[
        \bigcup V_x = U
      \]
      is open.
  \end{enumerate}
\end{proof}

\begin{definition}[Continuity]
   A map of topological spaces \( f: X \to Y \)
   is continuous if \( \forall V \subseteq Y\) open
   then
   \[
    {f}^{-1} (V) \subseteq X
   \]
   is open in \( X \).
\end{definition}

\begin{example}
  \( \text{id}: X \to X \) is cont. under the same topology.
\end{example}

\begin{example}
    \( f:X \to Y, f(x) = y \forall x \in X \).
    Let \( V \subseteq Y \). Then
    \[
      {f}^{-1} (V) = \begin{cases}
        X & y \in V \\
        \emptyset & y \notin V
      \end{cases}
    \]
    and since \( \emptyset, X \in \tau \), \( f \) is continuous.
\end{example}

\begin{example}
    \( f: X \to Y \), where \( X \) has the discrete topology.
    \( f \) is continuous since all \( {f}^{-1} (V) \) are open in \( X \).
\end{example}

\begin{example}
    \( f: X \to Y \), where \( Y \) has the indiscrete topology.
    \( f \) is continuous since \( {f}^{-1} (Y) = X \) and \( {f}^{-1} (\emptyset) = \emptyset \)
    are both opens.
\end{example}

\begin{nonexample}
  \( f:(\mathbb{R}, \tau_E) \to (\mathbb{R}, \tau_{dis}) \) is not continuous
  since \( {f}^{-1} (\{x\}) = \{x\} \) is not open wrt. \( \tau_E \).
\end{nonexample}

\begin{definition}[Coarser, finer]
  Let \( X \) be a set, and let \( \tau_1, \tau_2 \)
  be topologies on \( X \).
  \( \tau_1 \) is coarser than \( \tau_2 \) if \( \tau_1 \subset \tau_2  \).
  \( \tau_2 \) is finer than \( \tau_1 \) if \( \tau_1 \subset \tau_2  \).
\end{definition}
