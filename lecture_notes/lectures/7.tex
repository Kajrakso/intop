\section{Subspace topology, product topology, universal properties}
30.01

\begin{definition}[Subspace topology]
  Given \( A \subseteq X \), \( X \) topological space.
  We define the subspace topology by
  \[
    \tau_A = \{ U \cap A \mid U \text{ open in } X \} 
  \]
\end{definition}

\begin{proposition}
   Let \( A \subseteq X \), where \( X \) is a topological space
   with basis \( \mathscr{B} \). Then the collection
   \[
    \mathscr{B}_A = \{ B \cap A \mid B \in \mathscr{B}\} 
   \]
    is a basis for the subspace topology \( \tau_A \).
\end{proposition}

\begin{proof}
  We need to show that \( \mathscr{B}_A \) actually is a basis
  and then that the topology it generates coincides with the
  subspace topology.
 
  We show that we actually have a basis.
   \begin{enumerate}
     \item[B1)]
       Given \( x \in A \). In particular \( x \in X \),
       so there exists \( B \in \mathscr{B} \) such that
       \( x \in B \). Also, since \( x \in A \) we have that
       \( x \in B \cap A \in \mathscr{B}_A \).
     \item[B2)]
       Given \( x \in (B_1 \cap A) \cap (B_2 \cap A) = (B_1 \cap B_2) \cap A \).
       Since \( x \in B_1 \) and \( x \in B_2 \) we get \( B_3 \subseteq \mathscr{B} \)
       such that \( x \in B_3 \subseteq B_1 \cap B_2 \).
       Then
       \[
        x \in B_3 \cap A \subseteq (B_1 \cap B_2) \cap A = (B_1 \cap A) \cap (B_2 \cap A)
       \]
   \end{enumerate}

   Denote the topology generated by \( \mathscr{B}_A \) by \( \overline{\tau} \).
   We claim that \( \overline{\tau} = \tau_A \).
    Let us show that \( \overline{\tau} \subseteq  \tau_A \).
    Let \( U \subseteq A \) be open in \( \overline{\tau} \).
    Then \( U = \bigcup (B_i \cap A) \), where \( B_i \) open in \( X \).
    Hence \( B_i \cap A \) open in \( \tau_A \), so
    \( \overline{\tau} \subseteq \tau_A \).

    Next, show that \( \tau_A \subseteq \overline{\tau} \).
    Take \( U \cap A \in \tau_A \). Take \( x \in U \cap A \).
    Since \( U  \) open, there exists \( B \in \mathscr{B} \) such that
    \( x \in B \subseteq X \). Thus \( x \in B \cap A \subseteq U \cap A \).
\end{proof}

\begin{remark}
  Let \( A \subseteq X \), \( X \) topological space.
  Then the canonical inclusion \( \iota \) is continuous,
  where \( A \) has the subspace topology.
\end{remark}

\begin{theorem}
   Let \( X, Y \) be a topological spaces and let \( A \subseteq X \) be a subset of \( X \). Let \( f: Y \to A \) be a map, and let
   \( \iota: A \to X  \) be the inclusion map.
   Then the supspace topology on \( A \) is the \textit{unique} topology
   on \( A \) which satisfies the following
   \begin{equation} \label{eq:sub}
     f \text{ is continuous} \iff \iota \circ f \text{ is continuous.}
   \end{equation}
\end{theorem}
% https://q.uiver.app/#q=WzAsMyxbMCwyLCJZIl0sWzIsMiwiQSJdLFsyLDAsIlgiXSxbMCwxLCJmIl0sWzEsMiwiXFxpb3RhIiwyLHsic3R5bGUiOnsidGFpbCI6eyJuYW1lIjoiaG9vayIsInNpZGUiOiJ0b3AifX19XSxbMCwyLCJcXGlvdGEgXFxjaXJjIGYiXV0=
\[\begin{tikzcd}
	&& X \\
	\\
	Y && A
	\arrow["{\iota \circ f}", from=3-1, to=1-3]
	\arrow["f", from=3-1, to=3-3]
	\arrow["\iota"', hook, from=3-3, to=1-3]
\end{tikzcd}\]

\begin{proof}
  We first show that the subspace topology satisfies 
  \eqref{eq:sub} and then that it is the unique topology that
  satisfies \eqref{eq:sub}.

  Suppose \( f: Y \to A \) is continuous.
  Since \( \iota: A \to X \) is continuous, we get by
  composition that \( \iota \circ f \) is continuous.

  Suppose that \( \iota \circ f: Y \to X \) is continuous.
  Let \( U \subseteq X \) be a open subset of \( X \).
  Then 
  \[
    {f}^{-1} (U\cap A) = {f}^{-1}( {\iota}^{-1} (U))
    = {(\iota \circ f)}^{-1} (U)
  \]
  is open, so \( \iota \circ f \) is continuous.

  Suppose there exist another topology
  \( \hat{\tau} \) on \( A \) such that
  \eqref{eq:sub} holds. The following diagram commutes
  by \eqref{eq:sub}.
% https://q.uiver.app/#q=WzAsMyxbMCwyLCIoQSwgXFxoYXR7XFx0YXV9KSJdLFsyLDIsIihBLCBcXGhhdHtcXHRhdX0pIl0sWzIsMCwiWCJdLFswLDEsIlxcdGV4dHtpZH0iXSxbMSwyLCJcXGlvdGEiLDIseyJzdHlsZSI6eyJ0YWlsIjp7Im5hbWUiOiJob29rIiwic2lkZSI6InRvcCJ9fX1dLFswLDIsIiBcXGlvdGEgXFxjaXJjIFxcdGV4dHtpZH0iXV0=
\[\begin{tikzcd}
	&& X \\
	\\
	{(A, \hat{\tau})} && {(A, \hat{\tau})}
	\arrow["{ \iota \circ \text{id}}", from=3-1, to=1-3]
	\arrow["{\text{id}}", from=3-1, to=3-3]
	\arrow["\iota"', hook, from=3-3, to=1-3]
\end{tikzcd}\]
So, by continuity of the identity map,
\( {(\iota \circ \text{id})}^{-1} (U) = U \cap A \) is open in \( (A, \hat{\tau}) \). Hence \( \tau_A \subseteq  \hat{\tau} \).
Now, consider
% https://q.uiver.app/#q=WzAsMyxbMCwyLCIoQSwgXFx0YXVfQSkiXSxbMiwyLCIoQSwgXFxoYXR7XFx0YXV9KSJdLFsyLDAsIlgiXSxbMCwxLCJcXHRleHR7aWR9Il0sWzEsMiwiXFxpb3RhIiwyLHsic3R5bGUiOnsidGFpbCI6eyJuYW1lIjoiaG9vayIsInNpZGUiOiJ0b3AifX19XSxbMCwyLCJcXGlvdGEgXFxjaXJjIFxcdGV4dHtpZH0iXV0=
\[\begin{tikzcd}
	&& X \\
	\\
	{(A, \tau_A)} && {(A, \hat{\tau})}
	\arrow["{\iota \circ \text{id}}", from=3-1, to=1-3]
	\arrow["{\text{id}}", from=3-1, to=3-3]
	\arrow["\iota"', hook, from=3-3, to=1-3]
\end{tikzcd}\]
and note that \( \iota \circ \text{id}\) is continuous.
So \( \text{id} \) is continuous by \eqref{eq:sub},
and hence \( \hat{\tau} \subseteq \tau_A \).

We have shown that \( \hat{\tau} \subseteq \tau_A \)
and that \( \tau_A \subseteq  \hat{\tau} \) so \( \tau_A = \hat{\tau} \).
\end{proof}

\subsection{Product spaces}

\begin{proposition}
   \( X, Y  \) topological spaces. Then 
   \[
     \mathscr{B}_{X \times Y} = \{ U \times V \mid U \subseteq X \text{ open}, V \subseteq Y \text{ open} \} 
   \]
   is a basis for \( X \times Y \).
\end{proposition}

\begin{proof}
   We show the the two properties in the definition.
   \begin{enumerate}
     \item[B1)] Let \( (x, y) \in X \times Y \). Note that
       \( X \times Y \) is a basis element since
       \( X \) is open in \( X \) and \( Y \) is 
       open in \( Y \).
     \item[B2)] Let \( (U_1 \times V_1), (U_2 \times V_2) \)
       be basis elements and let 
       \[ 
        (x, y) \in (U_1 \times V_1) \cap (U_2 \times V_2).
       \]
       Then, the basis element \( (U_1 \cap U_2 ) \times (V_2 \cap V_2) \) is such that 
       \[
        (x, y)
          \in (U_1 \cap U_2 ) \times (V_2 \cap V_2)
          \subseteq (U_1 \times V_1) \cap (U_2 \times V_2)
       \]
       since \( (U_1 \cap U_2 ) \times (V_2 \cap V_2) =
          (U_1 \times V_1) \cap (U_2 \times V_2) \).
   \end{enumerate}
\end{proof}

\begin{definition}[Product Topology]
  Let \( X, Y \) be topological spaces. Then we define the
  product topology \( \tau_{X \times Y} \) to be
  the topology generated by \( \mathscr{B}_{X \times Y} \).
\end{definition}

\begin{proposition}
   Let \( X, Y \) be topological spaces.
Let \( \mathscr{B}_X, \mathscr{B}_Y \) be bases for \( X \) and \( Y \) respectivly.
Then the collection 
\[
  \mathscr{B}_X \times \mathscr{B}_Y = \{ B_X \times B_Y \mid B_X \in \mathscr{B}_X, B_Y \in \mathscr{B}_Y \} 
\]
is a basis for \( X \times Y \) that generates the product topology.
\end{proposition}

\begin{proof}
  We need to show that it is in fact a basis
  and then that is generates the product topology.
  \begin{enumerate}
      \item[B1)]
  Let \( (x, y) \in X \times Y \). In particular,
  \( x \in X \) and \( y \in Y \), so we get
  basis elements \( B_x \ni x, B_y \ni y \).
  Hence \( (x, y) \in B_x \times B_y \in \mathscr{B}_X \times \mathscr{B}_Y \).

  \item[B2)] Given basis elements
  \( B_{X}^1 \times B_{Y}^1, B_{X}^2 \times B_{Y}^2 \in
  \mathscr{B}_X \times \mathscr{B}_Y\).
  Let
  \[
    x \in (B_{X}^1 \times B_{Y}^1) \cap (B_{X}^2 \times B_{Y}^2) = 
(B_{X}^1 \cap B_X^2) \times (B_{Y}^1 \cap B_{Y}^2)
  \]
  Now, since \( \mathscr{B}_X, \mathscr{B}_Y \)
  are bases for \( X \) and \( Y \) respectivly
  we get \( B_X^3, B_Y^3 \) such that
  \[
    x \in B_X^3 \subseteq B_X^1 \cap B_X^2
  \]
  and
  \[
    y \in B_Y^3 \subseteq B_Y^1 \cap B_Y^2
  \]
  Hence
  \[
    (x, y) \in B_X^3 \times B_Y^3 \subseteq (B_X^1 \cap B_X^2) \times (B_Y^1 \cap B_Y^2) = 
  (B_{X}^1 \times B_{Y}^1) \cap (B_{X}^2 \times B_{Y}^2)
  \]
  \end{enumerate}

  Next, we show that it generates the product topology
  \( \tau_{X \times Y} \). We show that \( \tau_{\mathscr{B}_X \times \mathscr{B}_Y} \subseteq \tau_{X \times Y} \)
  and that 
\( \tau_{X \times Y} \subseteq \tau_{\mathscr{B}_X \times \mathscr{B}_Y} \).

  \begin{enumerate}
      \item[\( \subseteq \))]
        Take \( W \in \tau_{X \times Y} \).
        Then \( W = \bigcup_{i \in I} U_{X_i} \times V_{Y_i} \).
        Let \( (x, y) \in W \). Then \( \exists j \)
        such that \( (x, y) \in U_{X_j} \times V_{Y_j} \).
      So there exists basis elements \( B_{X_j},B_{Y_j} \)
      such that
      \[
        (x,y) \in B_{X_j} \times B_{Y_j} \subseteq U_{X_j} \times U_{Y_j}
      \]
      Hence \( \tau_{X \times Y} \subseteq \tau_{\mathscr{B}_X \times \mathscr{B}_Y}  \).
      \item[\( \supseteq \))]
        Take \( W \in \tau_{\mathscr{B}_X \times \mathscr{B}_Y} \). Then \( W = \bigcup_{i\in I} B_{X_i} \times B_{Y_i}  \), and since every \( B_{X_i}, B_{Y_i} \) are opens, we get
        that \( \tau_{\mathscr{B}_X \times \mathscr{B}_Y} \subseteq \tau_{X \times Y} \).
  \end{enumerate}
\end{proof}

\begin{remark}
    We have two canonical maps
    \begin{align*}
      \pi_X: X \times Y &\to X \\
      (x, y) &\mapsto x \\
      \pi_Y: X \times Y &\to Y \\
      (x, y) &\mapsto y
    \end{align*}

    Note, \( \pi_X \) is continuous, since
    for \( U \subseteq X \) open, then \( U \times Y \)
    open in \( X \times Y \). \( \pi_x(U \times Y ) = U \).
    Same for \( \pi_Y \).
\end{remark}
