\section{Compact spaces, product of compact spaces is compact.}

\begin{proposition}
    Let \( f: X \to Y \) be a continuous surjection.
    If \( X \) is compact, then \( Y \) is compact.
\end{proposition}

\begin{proof}
  Let \( \{ U_i \}_{i \in I}  \) be a cover of \( Y \).
  Then \( \{ {f}^{-1} (U_i) \}_{i \ in I} \) is a cover of
  \( X \) since \( f \) is continuous.
  Now, \( X \) is compact, so there exists a finite refinement
  of \( I \), say \( J \subseteq I \).
  Now \( \bigcup_{j \in J} U_j = Y \)  since
  \[
    Y = f(X) = f\left(\bigcup_{j \in J} {f}^{-1}(U_j)\right)
    = \bigcup_{j \in J} f\left({f}^{-1}(U_j)\right)
    = \bigcup_{j \in J} U_j
  \]
\end{proof}

\subsection{Product of compact spaces}

\begin{lemma}[Tubular neighborhood lemma]
   Let \( X, Y \) be topological spaces and let \( Y \) be compact.
   Fix \( x \in X \) such that \( U \subseteq X \times Y \) is a open subset of \( X \times Y \)
   and such that \( \{ x  \} \times Y \subseteq U  \).
   Then there exists an open subset \( W_x \subseteq X \) such that \( x \in W_x \) and \( W_x \times Y \subseteq U \).
\end{lemma}

\begin{proof}
  Since \( U \) is open, for all points \( (x, y) \),
  where \( x \) is the fiexd point from the statement,
  we have an open \( W_y \times V_y \subseteq U \).
  Then \( \{ V_y \} _{y \in Y} \) is a cover of 
  \( Y \). But \( Y \) is compact, so there exists
  a finite number of points \( y_0, \dots, y_n \) such that
  \[
  Y = \bigcup_{i \in I} V_{y_i}
  \]
  Let \( W_x = \bigcap_{i \in I} W_{y_i} \).
  \( W_x \) is open since \( I \) is finite.
  Now
  \[
    W_x \times Y \subseteq \bigcup_{i \in I} W_x \times V_{y_i} \subseteq U
  \]
\end{proof}

\begin{theorem}
  Let \( X, Y \) be compact topological spaces.
  Then \( X \times Y \) is compact.
\end{theorem}

\begin{proof}
  Let \( \{ U_i \}_{i \in I}  \) be a cover of
  \( X \times Y \).
  Pick \( x \in X \). Now \( \{ x \} \times Y \simeq Y \),
  and since \( Y \) compact and \( \{ U_i \}_{i \in I}  \)
  covers \( \{ x \} \times Y \)
  we have a finite refinement \( I_x \subseteq I \)
  such that \(\{ x \} \times Y \subseteq  \bigcup_{i \in I_x} U_i \).
  Define \( U_x = \bigcup_{i \in I_x} U_i \).

  By the Tubular neighborhood lemma there exists
  an open subset \( W_x \subseteq U_x \) such that
  \( x \in W_x \) and \( W_x \times Y \subseteq U_x \).

  \( x \) was arbitrary, so we get a cover of \( X \):
  \( \{ W_x \}_{x \in X}  \). \( X \) is compact
  so we pick a finite refinement \( j = 0, \dots, n \) such that
  \( \bigcup_{j = 0}^n W_{x_j} = X \).

  Then
  \[
    \bigcup_{j = 0}^n \bigcup_{i \in I_{x_j}} U_i
    = \bigcup_{j = 0}^n U_{x_j}
    \supseteq \bigcup_{j=0}^n W_{x_j} \times Y = X \times Y
  \] 
  And the double union on the left hand side
  of the equation is finite since it is a
  finite union of a finite union.
\end{proof}

\subsection{The closed interval is compact}

\begin{theorem}
  Let \( [a, b] \) ba a closed interval in \( \mathbb{R} \).
  Then \( [a, b] \) is compact.
\end{theorem}

\begin{proof}
    todo.
\end{proof}

\begin{definition}[boundedness]
    Let \( (X, d) \) be a metric space.
    Then \( A \subseteq X \) is bounded if there exists
    a constant \( L > 0 \) such that \( d(x, y) < L \)  for all \( x, y \in A \).
\end{definition}

\begin{theorem}[Heine-Borel]
  Let \( A \subseteq \mathbb{R}^n \) be a subspace of \( \mathbb{R}^n \).
  Then \( A \) is compact iff. \( A \) is closed and bounded.
\end{theorem}

\begin{proof}
    todo.
    idea: \( \Leftarrow \) by building a square around \( A \). square is product of closed intervals, which are compact.
\end{proof}

\begin{theorem}[Generalised extreme value theorem]
    Let \( f: X \to \mathbb{R} \) be a continuous map.
    If \( X \) is compact then there exist \( m, M \in X \)
    such that \( f(m) \le f(x) \le f(M) \) for all \( x \in X \).
\end{theorem}

\begin{proof}
   todo. 
\end{proof}

Now we can argue that \( f: [0, 1] \to S^1, t \mapsto (\cos 2\pi t, \sin 2\pi t) \) is a quotient map:

\begin{example}
  Let \( Z \subseteq [0, 1]  \) be a closed subset of \( [0, 1] \).
  Since \( [0, 1] \) is compact, \( Z \) is also compact.
  Now \( f \) is a continuous surjection so \( f(Z) \) is compact.
  \( S^1 \) is Hausdorff, so \( f(Z) \) has to be closed.
  Thus \( f \) is closed and \( f \) is thus a quotient map.
\end{example}

