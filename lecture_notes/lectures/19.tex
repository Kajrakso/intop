\section{Homotopy equivalences and the fundamental group}
20.03

\begin{definition}
    Let \( f,g: (X, x_0) \to (Y, y_0) \) be based maps
    We say that a homotopy
    \[
      H: I \times X \to Y
    \]
    is a based homotopy if \( H(t, x_0) = y_0 \)
    for all \( t \).
\end{definition}

\begin{lemma}
  \label{lma:based_maps_homotopic}
    Let \( f, g: (X, x_0) \to (Y, y_0) \)
    be based maps which are homotopic via
    a based homotopy. Then
    \[
      \pi_1 (f) = \pi_1(g)
    \]
\end{lemma}

\begin{proof}
    Let \( \gamma \in \pi_1(X, x_0) \).
    We need to show that \( f_*(\gamma) \simeq_p g_*(\gamma) \).
    That is, there exists a path homotopy
    for \( f \circ \gamma \) and \( g \circ \gamma \).
    Let \( H: I \times X \to X \) be a based homotopy
    for \( f \) and \( g \).
    Then
    \[
      \hat{H}: I \times I \xrightarrow{\text{id} \times \gamma}
      I \times X \xrightarrow{H} Y
    \]
    is this path homotopy.
    \begin{enumerate}
      \item \( \hat{H}(0, s) = H(0, \gamma(s)) = (f \circ \gamma) (s) \)
      \item \( \hat{H}(1, s) = H(1, \gamma(s)) = (g \circ \gamma) (s) \)
      \item \( \hat{H}(t, 0) = H(t, \gamma(0)) = y_0 \)
      \item \( \hat{H}(t, 1) = H(t, \gamma(1)) = )y_0 \)
    \end{enumerate}
\end{proof}

\begin{definition}[retract, retraction]
   Let \( A \subseteq X \) be a subspace.
   We say that \( A \) is a retract of \( X \)
   if there exists a map
   \[
    r: X \to A
   \]
   such that \( r \circ \iota = \text{id}_A \).
   We call \( r \) a retraction.
\end{definition}


\begin{example}
   Consider \( \mathbb{S}^1 \subseteq \mathbb{R}^2 \setminus \{ (0, 0) \}  \) .
   Then \( r(x) = x / \lvert \lvert x \rvert \rvert \) is a retraction.
\end{example}

\begin{lemma}
  \label{lma:retract->inj}
  Let \( x_0 \in A \subseteq X  \), and suppose
  that \( A \) is a retract of \( X \).
  Then the induced group homomorphism
  \[
    \pi_1(A, x_0) \longrightarrow \pi_1(X, x_0)
  \]
  is injective.
\end{lemma}

\begin{proof}
   Let \( r: X \to A \) be a retraction.
    Since \( x_0 \in A \) then \( r(x_0) = x_0 \).
    I get group homomorphisms
    \begin{equation}
      \pi_1(A, x_0) \xrightarrow{\pi_1(\iota)} \pi_1(X, x_0)
      \xrightarrow{\pi_1(r)} \pi_1(A, x_0)
    \end{equation}
    That compose to the identity: \( \pi_1(r) \circ \pi_1(\iota) = id \).
    Since the composite is injective, \( \pi_1(\iota) \) is injective.
\end{proof}

\begin{definition}[deformation retract]
    Let \( A \subseteq X \) be a subspace.
    A homotopy
    \[
      H: I \times X \to X
    \]
    is a deformation retract if the following holds
    \begin{enumerate}
      \item \( H(0, x) = x \)
      \item \( H(1, x) \in A \)
      \item \( H(t, a) = a, \forall a \in A \)
    \end{enumerate}
\end{definition}

\begin{remark}
   A deformation retract is a retract. 
\end{remark}

\begin{proof}
    Let \( A \subseteq X \) and \( H: I\times X \to X \)
    be a deformation retract. Then we get the retract
    \( r: X \to A \) by
    \[
      H(1, x): X \xrightarrow{r} A \xrightarrow{\iota} X
    \]

\end{proof}

\begin{theorem}
    Let \( x_0 \in A \subseteq X \) and suppose
    that \( A \) is a deformation retract of \( X \).
    Then we have a group isomorphism
    \[
      \pi_1(\iota): \pi_1(A, x_0) \to \pi_1(X, x_0)
    \]
\end{theorem}

\begin{proof}
  My notes here are hardly intelligible, so I provide
  my own proof. (I think this is what Fernando wrote).

  By lemma \ref{lma:retract->inj}
  \( \pi_1(\iota) \) is an injective group homomorphism.
  We need to show that it is surjective, i.e. that
  \( \pi_1(\iota) \circ \pi_1(r) = \text{id}  \).
  Let \( H: I \times X \to X \) be the
  deformation retract.
  Let \( H(1, x) = r: X \to A \) denote the retract.
  Now, \( H \) is a based homotopy of \( \text{id} \) and
  \( r \circ \iota  \):
  \begin{align*}
    H(t, x_0) &= x_0 & \text{since \( x_0 \in A \). } \\
    H(0, x) &= x \\
    H(1, x) &= \left(\iota \circ r\right)(x)
  \end{align*}
  By lemma \ref{lma:based_maps_homotopic}
  \( \pi_1(\iota) \circ \pi_1(r) = \text{id} \),
  so \( \pi_1(\iota) \) is surjective.

  Hence, \( \pi_1(\iota) \) is an isomorphism
  of groups.
\end{proof}

\begin{exercise}
    Show that \( \mathbb{S}^1 \subseteq \mathbb{R}^2 \setminus \{ (0, 0) \}  \)
    is a deformation retract.
\end{exercise}

\begin{definition}[homotopy equivalence]
    A map \( f: X \to Y \) is a homotopy equivalence
    if there exists a continuous map
    \[
      g: Y \to X
    \]
    such that \( g \circ f \sim \text{id}_X \)
    and \( f \circ g \sim \text{id}_Y \).
\end{definition}

\begin{example}
  Let \( A \subseteq X \) be a deformation retract.
  Then \( \iota: A \to X \) is a homotopy equivalence.
\end{example}

\begin{example}
    \( 0 \in \mathbb{R}^n \). \( \mathbb{R}^n \sim * \).
\end{example}

\begin{definition}[Contractible space]
   A space is said to be contractible
   if it is homotopy equivalent to
   the point space.
\end{definition}

\begin{definition}[homotopy type]
    We say that \( X \) and \( Y \)
    have the same homotopy type if \( X \)
    is homotopy equivalent to \( Y \).
    Notation:
    \[
      [X] = \{ Y \text{top. space} \mid X \text{ homotopic equivalent to } Y \} 
    \]
\end{definition}

\begin{example}
  \( [\mathbb{R}^n] = [*] \).
\end{example}

\begin{lemma}
  \label{lma:hom_diag_lemma}
    Let \( f, g: X \to Y \) be maps.
    Suppose \( H \) is a homotopy
    \[
      H: I \times X \to Y
    \]
    between \( f, g \).
    Given \( x_0 \in X \), and let
    \[
      \alpha = H(\cdot, x_0): I \to X
    \]
    Then we have a commutative diagram 
    of groups:

    % https://q.uiver.app/#q=WzAsMyxbMCwwLCJcXHBpXzEoWCwgeF8wKSJdLFsyLDIsIlxccGlfMShZLCBnKHhfMCkpIl0sWzIsMCwiXFxwaV8xKFksIGYoeF8wKSkiXSxbMCwxLCJcXHBpXzEoZykiLDJdLFswLDIsIlxccGlfMShmKSJdLFsyLDEsIlxcaGF0e1xcYWxwaGF9Il1d
\[\begin{tikzcd}
	{\pi_1(X, x_0)} && {\pi_1(Y, f(x_0))} \\
	\\
	&& {\pi_1(Y, g(x_0))}
	\arrow["{\pi_1(f)}", from=1-1, to=1-3]
	\arrow["{\pi_1(g)}"', from=1-1, to=3-3]
	\arrow["{\hat{\alpha}}", from=1-3, to=3-3]
\end{tikzcd}\]
where \( \hat{\alpha}([\gamma]) = [\alpha * \gamma * \overline{\alpha}] \).
\end{lemma}

\begin{proof}
   We need to show that
   \( \hat{\alpha} \circ \pi_1(f) = \pi_1(g) \).
   That is;
   \[
     [\alpha * (f \circ \gamma) * \overline{\alpha}]
     = [g \circ \gamma]
   \]
   for all \( \gamma \in \pi_1(X, x_0) \).
  Let \( \gamma  \) be a loop based at \( x_0 \).
  Define
  \begin{equation}
      H'(s, t) = \begin{cases}
        \overline{\alpha}(4s) & 0 \le s \le t/4 \\
        H(1-t, \gamma(\frac{4s - t}{4 - 3t})) & t/4 \le s \le 1 - t/2 \\
        \alpha(2s-1) & 1-t/2 \le s \le 1
      \end{cases}
  \end{equation}
  Check that \( H' \) is a path homotopy of 
  \( \alpha * (f \circ \gamma) * \overline{\alpha} \)
  and \( g \circ \gamma \):
      \begin{align*}
        H'(0, t) = \overline{\alpha}(0) 
                 = H(1, x_0) 
                 = g(x_0)
      \end{align*}
      \begin{align*}
        H'(1, t) = \alpha(1)
                 = H(1, x_0)
                 = g(x_0)
      \end{align*}
      \begin{align*}
        H'(s, 0) =  \begin{cases}
            \overline{\alpha} (4s) & 0 \le s \le 0 \\
            H(1, \gamma(s)) & 0 \le s \le 1 \\
            \alpha(2s - 1) & 1 \le s \le 1
        \end{cases} 
                 = H(1, \gamma(s))
                 = (g \circ \gamma) (s)
    \end{align*}
      \begin{align*}
        H'(s, 1) &= \begin{cases}
            \overline{\alpha} (4s) & 0 \le s \le 1/4 \\
            H(0, \gamma(4s - 1)) & 1/4 \le s \le 1/2 \\
            \alpha(2s - 1) & 1/2 \le s \le 1
          \end{cases} \\
          &= \begin{cases}
            \overline{\alpha} (4s) & 0 \le s \le 1/4 \\
            f\left(\gamma(4s - 1)\right) & 1/4 \le s \le 1/2 \\
            \alpha(2s - 1) & 1/2 \le s \le 1
          \end{cases}
          = (\alpha * (f \circ \gamma) * \overline{\alpha}) (s)
      \end{align*}
\end{proof}

\begin{theorem}
    Let \( f: X \to Y \) be a homotopy equivalence.
    Then the map \[
      \pi_1(f): \pi_1(X, x_0) \to \pi_1(Y, f(x_0))
    \]
    is a group isomorphism.
\end{theorem}

\begin{proof}
  Since \( f \) is a homotopy equivalence
  there exists a map \( g: Y \to X \)
  and a homotopy between
  \( g \circ f \) and \( \text{id}_X \).
  By lemma \ref{lma:hom_diag_lemma}
  we get the following diagram
    % https://q.uiver.app/#q=WzAsMyxbMCwwLCJcXHBpXzEoWCwgeF8wKSJdLFsyLDAsIlxccGlfMShYLCAoZyBcXGNpcmMgZikoeF8wKSkiXSxbMiwyLCJcXHBpXzEoWCwgeF8wKSJdLFswLDEsIlxccGlfMShnIFxcY2lyYyBmKSJdLFswLDIsIlxccGlfMShcXHRleHR7aWR9KSIsMl0sWzEsMiwiXFxoYXR7XFxhbHBoYX0iXV0=
  \[\begin{tikzcd}
    {\pi_1(X, x_0)} && {\pi_1(X, (g \circ f)(x_0))} \\
    \\
    && {\pi_1(X, x_0)}
    \arrow["{\pi_1(g \circ f)}", from=1-1, to=1-3]
    \arrow["{\text{id}}"', from=1-1, to=3-3]
    \arrow["{\hat{\alpha}}", from=1-3, to=3-3]
  \end{tikzcd}\]
  since \( \hat{\alpha} \circ \pi_1(g) \circ \pi_1(f) = \text{id}\),
  \( \pi_1(f) \) is injective.
  Next, since \( f \circ g \sim \text{id}_Y \)
  we get
    % https://q.uiver.app/#q=WzAsMyxbMCwwLCJcXHBpXzEoWSwgZih4XzApKSJdLFsyLDAsIlxccGlfMShZLCAoZiBcXGNpcmMgZyBcXGNpcmMgZikoeF8wKSkiXSxbMiwyLCJcXHBpXzEoWSwgeF8wKSJdLFswLDEsIlxccGlfMShmIFxcY2lyYyBnKSJdLFswLDIsIlxcdGV4dHtpZH0iLDJdLFsxLDIsIlxcaGF0e2V9Il1d
    \[\begin{tikzcd}
      {\pi_1(Y, f(x_0))} && {\pi_1(Y, (f \circ g \circ f)(x_0))} \\
      \\
      && {\pi_1(Y, x_0)}
      \arrow["{\pi_1(f \circ g)}", from=1-1, to=1-3]
      \arrow["{\text{id}}"', from=1-1, to=3-3]
      \arrow["{\hat{e}}", from=1-3, to=3-3]
    \end{tikzcd}\]

    Here my notes are not easy to decode,
    but I think the proof goes as follows:

    So \( \hat{e} \circ \pi_1(f) \circ \pi_1(g) = \text{id} \),
    hence \( \hat{e} \circ \pi_1(f) \) is surjective.
    Since \( \hat{e} \) is just conjugation by a path
    \( e \), \( \pi_1(f) \) is surjective.

    Hence, \( \pi_1(f) \) is a bijection, 
    and thus an isomorphism.
\end{proof}

