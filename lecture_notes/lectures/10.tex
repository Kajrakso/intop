\section{Quotients, open maps, universal property of the quotient topology}
07.02

\begin{definition}[Equivalence relation]
   Let \( X \) be a set. Then \( R \subseteq X \times X \) 
   is an equivalence relation if
   \begin{enumerate}
     \item \( (x, x) \in R \, \forall x \in X \)
     \item \( (x, y) \in R \implies (y, x) \in R \)
     \item \( (x, y) \in R, (y, z) \in R \implies (x, z) \in R \)
   \end{enumerate}
\end{definition}

\begin{definition}[Equivalence class]
   The equivalence class of \( x \)
   in a set \( X \) equipped with equivalence relation
   \( \sim \) is defined as
   \[
     [x] = \{ y \in X \mid x \sim y \} 
   \]
\end{definition}

\begin{lemma}
    Let \( X \) be a set equipped
    with equivalence relation \( \sim \).
    Then
    \[
      [x] = [y] \iff x \sim y
    \]
\end{lemma}

\begin{proof}
  If \( [x] = [y] \), then \( y \in [x] \) so \( y \sim x \), which proves one direction.
  Assume now that \( x \sim y \). Pick some \( z \in [y] \).
  Then \( y \sim z \) by the definition of the equivalence class.
  Use the assumption and the transitivity of the equivalence relation to arrive at
  \[
    x \sim y, y \sim z \implies x \sim z.
  \]
  So \( z \in [x] \), and since \( z \) was arbitrary, \( [y] \subseteq [x] \). The other inclusion is similar.
\end{proof}

\begin{definition}
  Let \( X \) be a set and let \( {\sim} \) be an equivalence relation
    on \( X \). Denote by \( X /{\sim} \) the set of equivalence classes:
    \[
      X /{\sim} = \left\{ [x] \mid x \in X \right\} 
    \]
\end{definition}

\begin{proposition}
   Let \( X, Y \) be sets, and let \( \sim \) be an equivalence relation
   on \( X \).
   Given \( f: X \to  Y \), then \( f \) factors through
   \( X / {\sim} \) (the diagram commutes) iff. \( \forall x, y \in X \) we have that
   \( x \sim y \implies f(x) = f(y) \).

% https://q.uiver.app/#q=WzAsMyxbMCwwLCJYIl0sWzIsMCwiWSJdLFswLDIsIlgve1xcc2ltfSJdLFswLDEsImYiXSxbMCwyLCJcXHBpIiwyXSxbMiwxLCJcXG92ZXJsaW5le2Z9IiwyLHsic3R5bGUiOnsiYm9keSI6eyJuYW1lIjoiZGFzaGVkIn19fV1d
\[\begin{tikzcd}
	X && Y \\
	\\
	{X/{\sim}}
	\arrow["f", from=1-1, to=1-3]
	\arrow["\pi"', from=1-1, to=3-1]
	\arrow["{\overline{f}}"', dashed, from=3-1, to=1-3]
\end{tikzcd}\]
\end{proposition}

\begin{proof}
  Assume that \( f \) factors through \( X / {\sim} \).
  If \( x \sim y \), so that \( \pi(x) = \pi(y) \)
  we get
  \[
    f(x) = \overline{f}(\pi(x)) = \overline{f}(\pi(y)) = f(y).
  \]

  For the other direction, define \( \overline{f}([x]) = f(x) \).
  This mapping is well defined since if \( [x] = [y] \) then
  \( x \sim y \), and so \( f(x) = f(y) \) by assumption.
  We see that \( f  \) factors through \( X /{\sim} \) since
  \( f(x) = \overline{f}([x]) = \overline{f}(\pi(x)) \).
\end{proof}

\begin{definition}[quotient topology]
   Let \( X \) be a toplogical space.
   Let \( \pi:X \to A \) be a surjection.
   We define define a topology on \( A \) by 
   declaring \( U \subseteq A \) to be open
   if \( {\pi}^{-1} (U) \) is open in \( X \).
\end{definition}

\begin{definition}[open and closed maps]
   A map \( f:X \to Y  \) of topological spaces \( X \) and \( Y \)
   is said to be open if \( \forall U \subseteq X \) open then \( f(U) \) open in \( Y \). The map is said to be closed if 
\( \forall V \subseteq X \) closed then \( f(V) \) closed in \( Y \).
\end{definition}

\begin{theorem}
    Let \( f:X \to Y \) be a cont. bijection of topological spaces.
    TFAE:
    \begin{enumerate}
      \item \( f \) is a homeomorphism.
      \item \( f \) is open.
      \item \( f \) is closed.
    \end{enumerate}
\end{theorem}

\begin{proof}
   Sps. \( f \) is a homeomorphism.
   Then \( f \) is trivially open and closed,
   since \( {f}^{-1}  \) is continuous.

   Sps. \( f \) is open. Since \( f \) is bij.,
   there exists a inverse map \( g = {f}^{-1}  \).
   Take \( U \subseteq X \) open. Then \( f(U) = {g}^{-1} (U) \)
   is open in \( Y \), so \( g \) is continuous. Hence
   \( f \) is a homeomorphism.

   Similar proof for when \( f \) is closed.
\end{proof}

\begin{theorem}
  \label{thm:cont_surj_closed->quotient}
    Let \( \pi : X\to A    \) be a continuous surjection.
    Then
    \begin{enumerate}
      \item \( \pi \) is open \( \implies \)  \( \pi \) is a quotient map.
      \item \( \pi \) is closed \( \implies \)  \( \pi \) is a quotient map.
    \end{enumerate}
\end{theorem}

\begin{proof}
   Suppose \( \pi \) is open.
   We need to show that \( U \subseteq A \) is open iff. 
   \( {\pi}^{-1}(U)  \) is open in \( X \).
    \begin{enumerate}
      \item[\( \Rightarrow \))]
        Given \( U \subseteq A \) is open, then 
   \( {\pi}^{-1}(U)  \) is open since \( \pi \) is continuous.
      \item[\( \Leftarrow \))]
        Given \( {\pi}^{-1}(U)  \subseteq X \) open.
        Then \( \pi( {\pi}^{-1} (U) ) = U \) since \( \pi \) 
        surjective. U is open
        since \( \pi \) is an open map.
    \end{enumerate}

    Same for \( \pi \) closed.
\end{proof}

\begin{theorem}
   Let \( X \) be a topological space.
   Let \( \pi: X \to A \) be a continuous surjection.
   Then the quotient topology on \( A \)
   is the unique topology on \( A \)
   satisfying the universal property

   \begin{equation}
   g:A \longrightarrow Y \text{ cont.}
   \iff
% https://q.uiver.app/#q=WzAsMyxbMCwwLCJYIl0sWzAsMiwiQSJdLFsyLDAsIlkiXSxbMCwxLCJcXHBpIiwyLHsic3R5bGUiOnsiaGVhZCI6eyJuYW1lIjoiZXBpIn19fV0sWzAsMiwiZiJdLFsxLDIsIlxcb3ZlcmxpbmV7Zn0iLDJdXQ==
\begin{tikzcd}
	X && Y \\
	\\
	A
	\arrow["f", from=1-1, to=1-3]
	\arrow["\pi"', two heads, from=1-1, to=3-1]
	\arrow["{g}"', from=3-1, to=1-3]
\end{tikzcd}
  g \circ \pi \text{ cont.}
  \end{equation}
\end{theorem}

\begin{proof}
  We show both directions, and then uniqueness.

   \begin{enumerate}
   \item[\( \Rightarrow \))] Assume \( g \) is cont.
     Since \( \pi \) is cont.
     by construction and composition of cont. functions
     is cont., then so is \( g \circ \pi \).
   \item[\( \Leftarrow \))] Assume \( g \circ \pi \) is cont..
     Take \( U \subseteq Y \) open. Then
     \[
      {(g \circ \pi)}^{-1} (U)  = {\pi}^{-1} ({g}^{-1}(U) )
      \]
     is open since \( g \circ \pi \) continuous.
     Since \( \pi \) is a quotient map, \( {g}^{-1} (U) \)
     is open, so \( g \) is continuous.
   \end{enumerate}

  Let \( \tau_q \) denote the quotient topology on \( A \).
  Assume there exists another topology \( \hat{\tau} \) on \( A \)
  such that the universal property holds.
  Consider
  
% https://q.uiver.app/#q=WzAsMyxbMCwwLCIoWCwgXFx0YXVfWCkiXSxbMCwyLCIoQSwgXFxoYXR7XFx0YXV9KSJdLFsyLDAsIihBLCBcXGhhdHtcXHRhdX0pIl0sWzAsMSwiXFxwaSciXSxbMCwyLCJcXHBpJyIsMl0sWzEsMiwiXFx0ZXh0e2lkfSJdXQ==
\[\begin{tikzcd}
	{(X, \tau_X)} && {(A, \hat{\tau})} \\
	\\
	{(A, \hat{\tau})}
	\arrow["{\pi'}"', from=1-1, to=1-3]
	\arrow["{\pi'}", from=1-1, to=3-1]
	\arrow["{\text{id}}", from=3-1, to=1-3]
\end{tikzcd}\]

  The identity map is continuous, and since
  \( \hat{\tau} \) satisfies the universal property
  \( \text{id} \circ \pi' = \pi'  \) is continuous.
  So a open subset \( U \subseteq A \) gives \( {\pi'}^{-1}(U)  \)
  open. Hence \(hat{\tau} \subseteq \tau_q \).

  Now, consider the diagram
% https://q.uiver.app/#q=WzAsMyxbMCwwLCIoWCwgXFx0YXVfeCkiXSxbMCwyLCIoQSwgXFxoYXR7XFx0YXV9KSJdLFsyLDAsIihBLCBcXHRhdV9xKSJdLFswLDEsIlxccGknIl0sWzEsMiwiXFx0ZXh0e2lkfSJdLFswLDIsIlxccGkiLDJdXQ==
\[\begin{tikzcd}
	{(X, \tau_x)} && {(A, \tau_q)} \\
	\\
	{(A, \hat{\tau})}
	\arrow["\pi"', from=1-1, to=1-3]
	\arrow["{\pi'}", from=1-1, to=3-1]
	\arrow["{\text{id}}", from=3-1, to=1-3]
\end{tikzcd}\]
Both \( \pi \) and \( \pi' \) are continuous by assumption.
Since \( \hat{\tau} \) is such that the universal property is satisfied
\( \text{id} \) has to be continous. Thus \( \tau_q \subseteq \hat{\tau} \).
\end{proof}

\subsection{Connected topological spaces}

\begin{definition}[separation]
    Let \( X \) be a topological space.
    A pair of opens \( U, V \subseteq X \)
    such that \( U, V \neq \emptyset, X \) form 
    a separation of \( X \) if
    \begin{enumerate}
      \item \( U \cup V = X \)
      \item \( U \cap V = \emptyset \)
    \end{enumerate}
\end{definition}

\begin{definition}[connectivity]
   A topological space \( X \) is connected
   if no separation exists. \( X \) is disconnected
   if it is not connected.
\end{definition}


